\documentclass{article}
\usepackage{graphicx} 
\usepackage{fancyhdr}
\usepackage{titling}
\usepackage{longtable}
\usepackage{hyperref}


\title{AI-Powered University Information System \\ \large Software Requirement Specification (SRS)}
\author{Jasmine Morales \and Tiana Luu \and Kira Ma \and Linda Tien}
\date{December 2025}

\setlength{\droptitle}{6cm}

\pagestyle{fancy}
\fancyhead[L]{AI-Powered University System SRS}
\fancyhead[R]{Page \thepage}
\fancyfoot[C]{}

\begin{document}

\begin{titlepage}
\maketitle
\thispagestyle{empty}
\end{titlepage}

\thispagestyle{empty}
\tableofcontents
\newpage

\section*{Version History}
\addcontentsline{toc}{section}{Version History}
\begin{table}[ht]
    \centering
    \begin{tabular}{|c|c|c|c|}
    \hline
    \textbf{User} & \textbf{Date} & \textbf{Reason for Changes} & \textbf{Version}\\
    \hline
        Jasmine Morales & 11/13/25 & Initial draft for Snapshot 1 & 1.0\\
    \hline
        Tiana Luu & 11/20/25 & Added interface requirements & 1.1\\
    \hline
        Kira Ma & 11/22/25 & Expanded introduction and overview & 1.2\\
    \hline
        Linda Tien & 11/30/25 & Added legal/ethical considerations & 1.3\\
    \hline
    \end{tabular}
\end{table}
\newpage

\section{Introduction}
\subsection{Purpose}
The purpose of this document is to provide a detailed description of the AI-Powered University Information System. It outlines the requirements, features, and functionality of the software to ensure clarity for developers, stakeholders, and users.
\subsection{Intended Audience}
This document is intended for software developers, project managers, data analysts, university administrators, and any stakeholders interested in understanding the system requirements.
\subsection{Overview}
The AI-Powered University System will be a web-based application designed to help students quickly access academic information such as course details, schedules, faculty contacts, and campus resources. The system leverages AI to provide personalized recommendations and streamline information retrieval.
\newpage

\section{External Interface Requirements}
\subsection{User Interfaces}
\begin{itemize}
    \item The application will include a responsive web-based interface accessible across devices (desktop, tablet, mobile).
    \item The interface will be intuitive and user-friendly, with clear navigation menus and search functionality.
    \item Accessibility features will be included to support users with disabilities (e.g., screen reader compatibility).
\end{itemize}

\subsection{Software Interfaces}
\begin{itemize}
    \item The system will integrate with Microsoft SQL Server for database management.
    \item APIs will be used to connect with existing university systems (e.g., student records, course catalog).
    \item The interface layer will be implemented using JavaFX for modularity and scalability.
\end{itemize}
\newpage

\section{Legal and Ethical Considerations}
\begin{itemize}
    \item \textbf{Data Privacy:} All student and faculty data will be stored securely, following FERPA and GDPR guidelines.
    \item \textbf{User Consent:} Users must provide consent before their data is used for AI-driven recommendations.
    \item \textbf{Data Protection:} Encryption will be applied to sensitive information such as student IDs, grades, and personal details.
    \item \textbf{Ethical Use of AI:} The recommendation engine will avoid bias and ensure fairness in information delivery.
\end{itemize}
\newpage

\section{Glossary}
\begin{longtable}{|c|p{10cm}|}
\hline
\textbf{Acronym} & \textbf{Definition} \\
\hline
UI & User Interface \\
\hline
API & Application Programming Interface \\
\hline
DB & Database \\
\hline
GDPR & General Data Protection Regulation \\
\hline
FERPA & Family Educational Rights and Privacy Act \\
\hline
AI & Artificial Intelligence \\
\hline
\end{longtable}
\newpage

\section{References}
\begin{itemize}
    \item Microsoft SQL Server Documentation: \url{https://learn.microsoft.com/sql}
    \item JavaFX Official Documentation: \url{https://openjfx.io}
    \item GDPR Guidelines: \url{https://gdpr-info.eu}
    \item FERPA Regulations: \url{https://studentprivacy.ed.gov/ferpa}
\end{itemize}


\end{document}

