\documentclass[11pt]{article}

% Packages
\usepackage[margin=1in]{geometry}
\usepackage{hyperref}
\usepackage{longtable}
\usepackage{graphicx}
\usepackage{array}

\hypersetup{
  colorlinks=true,
  linkcolor=blue,
  urlcolor=blue
}

%----------------- Cover Page Info -----------------
\title{Software Design Document (SDD) \\ Version 1.0}
\author{Group 12}
\date{December 5, 2025}

\begin{document}

%----------------- Cover Page -----------------
\maketitle
\thispagestyle{empty}
\clearpage

%----------------- Table of Contents -----------------
\pagenumbering{roman}
\tableofcontents
\clearpage

%----------------- Version Description -----------------
\section*{Version Description}
\addcontentsline{toc}{section}{Version Description}

\begin{center}
\begin{tabular}{|m{0.2\textwidth}|m{0.55\textwidth}|m{0.2\textwidth}|}
\hline
\textbf{Version} & \textbf{Description} & \textbf{Date} \\
\hline
1.0 & Initial SDD for Snapshot 1, including system overview, architecture, UI description, and database design. & <Date> \\
\hline
\end{tabular}
\end{center}

\clearpage
\pagenumbering{arabic}

%----------------- 1. Introduction -----------------
\section{Introduction}

\subsection{Purpose}
Describe the purpose of this Software Design Document and how it will be used by the team and stakeholders.

\subsection{Intended Audience}
Describe who should read this document (developers, testers, product owner, instructors, etc.) and what each group should gain from it.

\subsection{System Overview}
Provide a high-level overview of the system: main features, target users, and the context in which it will be used.

%----------------- 2. System Architecture -----------------
\section{System Architecture}

\subsection{Workflow of the System}
Describe the overall workflow from the user’s perspective and the internal flow between components. Optionally refer to diagrams you create (e.g., sequence diagrams, flow diagrams).

\subsection{Component Breakdown}
Describe the main components, such as:
\begin{itemize}
  \item Server-side (e.g., REST API, business logic, services)
  \item Client-side (e.g., web front-end, mobile app views)
  \item Data storage (e.g., relational database, ORM layer)
  \item External services or APIs (if any)
\end{itemize}

Explain how these components interact and the responsibilities of each.

%----------------- 3. User Interface -----------------
\section{User Interface}

\subsection{How to Use the System}
Describe user-facing flows:
\begin{itemize}
  \item How users sign up / log in (if applicable)
  \item How they perform key tasks
  \item Navigation structure between pages or screens
\end{itemize}

\subsection{Database Design and Explanation}
Describe the database design:
\begin{itemize}
  \item List main entities/tables and their purpose
  \item Explain important relationships (one-to-many, many-to-many, etc.)
\end{itemize}

You can include an ER diagram or schema diagram as a figure.

\subsection{Screenshots (Optional)}
If you already have UI mockups or early screenshots, include them here:
\begin{figure}[h]
  \centering
  \includegraphics[width=0.8\textwidth]{images/sample-screen.png}
  \caption{Sample UI screen}
  \label{fig:sample-ui}
\end{figure}

%----------------- 4. Glossary -----------------
\section{Glossary}

\begin{center}
\begin{longtable}{|m{0.25\textwidth}|m{0.65\textwidth}|}
\hline
\textbf{Acronym} & \textbf{Definition} \\
\hline
UI & User Interface \\
\hline
API & Application Programming Interface \\
\hline
DB & Database \\
\hline
HTTP & Hypertext Transfer Protocol \\
\hline
% Add more as needed
\end{longtable}
\end{center}

%----------------- 5. References -----------------
\section{References}

List any documentation, books, tutorials, or web resources that helped you design or implement the system. Use a simple itemized list or a bibliography environment, for example:
\begin{itemize}
  \item Overleaf LaTeX documentation and templates. \url{https://www.overleaf.com/latex/templates}
  \item Any course notes or assignment descriptions (cite per your course policy).
  \item API documentation or framework docs (e.g., Spring Boot, React, etc.).
\end{itemize}

\end{document}\documentclass{article}
\usepackage{graphicx} % Required for inserting images

\title{final}
\author{Tiana Luu}
\date{December 2025}

\begin{document}

\maketitle

\section{Introduction}

\end{document}
